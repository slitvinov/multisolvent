\subsection{colors}\label{colors}

Colors are activated by \texttt{\#define\ multi\_solvent\ true}.

Blue: represents outer solvent particles Red: represents inner solvent
particles

Color is an integer field. It is packed into \texttt{cloud.md} along
with \texttt{pp} arrays. Interactions (fsi, dpdl, dpdr, wall) get
\texttt{forces::Pa} from Clouds:

\begin{verbatim}
namespace forces {
struct Pa {
    float x, y, z, vx, vy, vz;
    int kind;
    int color;
};
\end{verbatim}

and passes it to \texttt{forces::gen}. See \texttt{src/forces/imp.h} which
decides which pair force to apply based on \texttt{int\ color}. It knows
about

\begin{verbatim}
enum {BLUE_COLOR, RED_COLOR};
\end{verbatim}

And miss-uses \texttt{gammadpd\_solv}, \texttt{gammadpd\_rbc} and
\texttt{gammadpd\_wall} to set DPD parameters for blue-blue, red-red,
and red-blue interactions.

In \texttt{sim::} colors are stored in

\begin{verbatim}
flu::QuantsI     qc;
\end{verbatim}

where

\begin{verbatim}
struct QuantsI {
    int *ii, *ii0;
    int *ii_hst;
 };
\end{verbatim}

in \texttt{sim::sim\_gen()} gets colors by calling

\begin{verbatim}
flu::gen_quants(&o::q, &o::qc);`
\end{verbatim}

which has a part different in different \texttt{u.md}. Hard-coded
color schemes are

\begin{itemize}
\tightlist
\item
  \texttt{flu/\_ussr} : all \texttt{BLUE\_COLOR} (default)
\item
  \texttt{flu/\_zurich} : flag of zurich in XY
\item
  \texttt{flu/\_bangladesh} : a sphere in the center
\item
  \texttt{flu/\_france} : flag of france in XY
\end{itemize}

Example \url{run/color/run}

\subsection{recolor}\label{recolor}

Re-coloring is done every \texttt{freq\_color} timesteps.
\texttt{freq\_color=0} means no re-coloring.
