\subsection{Introduction}\label{introduction}

\texttt{file.bop} format (ascii, \texttt{\textless{}N\textgreater{}} is
the number of particles):

\begin{verbatim}
 <N>
 DATA_FILE: <file.values>
 DATA_FORMAT: <float|ascii|double|int|iascii>
 VARIABLES: <x> <y> <z> <vx> <vy> <vz> <id> ...
\end{verbatim}

\texttt{file.values} format:

\begin{verbatim}
x[0] y[0] z[0] vx[0] vy[0] vz[0] id[0]
...

x[N-1] y[N-1] ...
\end{verbatim}

The \texttt{ascii} format is assumed to be single precision floating
points data.\\
Th \texttt{iascii} format is the ascii version of integer data.

\subsection{Installation}\label{installation}

The following will install the binaries into \texttt{\$\{HOME\}/bin}.
This folder needs to be in the \texttt{PATH} variable. Furthermore, it
installs the headers and library into
\texttt{\$\{HOME\}/prefix/bop/include} and
\texttt{\$\{HOME\}/prefix/bop/lib}, respectively.

\begin{verbatim}
make && make install
\end{verbatim}

\subsection{Usage}\label{usage}

Convert bop files \texttt{in1.bop}, \texttt{in2.bop}, \ldots{},
\texttt{inN.bop} into a single vtk file \texttt{out.vtk}:

\begin{verbatim}
bop2vtk out.vtk in1.bop in2.bop ... inN.bop
\end{verbatim}

Dump ascii data from bop files \texttt{in1.bop}, \texttt{in2.bop},
\ldots{}, \texttt{inN.bop} into \texttt{out.txt}:

\begin{verbatim}
bop2txt in1.bop in2.bop ... inN.bop > out.txt
\end{verbatim}

\subsection{Testing}\label{testing}

Uses \texttt{atest} framework (https://gitlab.ethz.ch/mavt-cse/atest)

\begin{verbatim}
atest bop2txt.cpp
atest bop2vtk.cpp
\end{verbatim}
