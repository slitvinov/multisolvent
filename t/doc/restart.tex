Restart stores simulation variables of solvent, wall, rbcs and rigid
bodies under the following file naming:

\begin{verbatim}
strt/[code]/[XXX].[YYY].[ZZZ]/[ttt].[ext]
\end{verbatim}

where: * \texttt{{[}code{]}} is \texttt{flu} (solvent), \texttt{wall},
\texttt{rbc} or \texttt{rig} (rigid bodies) *
\texttt{{[}XXX{]}.{[}YYY{]}.{[}ZZZ{]}} are the coordinates of the
processor (no dir if single processor) * \texttt{{[}ttt{]}} is the id of
the restart; it has a magic value \texttt{final} for the last step of
simulation and is used to start from there. * \texttt{{[}ext{]}} is the
extension of the file: \texttt{bop} for particles, \texttt{ss} for
solids, \texttt{id.bop} for particle ids

Special case:

\begin{verbatim}
strt/[code]/[magic name].[ext]
\end{verbatim}

example: template frozen particles from rigid bodies or walls:
\texttt{templ}
