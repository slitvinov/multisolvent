The benefits of the restartable simulation are huge: computational
time is not wasted on transient part of the simulation, external and
not nececarily high performance tools can be used to
set up \texttt{uDeviceX} simulations, multi-stage simulations with very
different govering equations at every stage are possible, debuging and
testing is much simpler.

Restart is a dump of the state of \texttt{uDeviceX} simulation:
position of every solvent, wall, and RBC particles are dumped. State
of solid objects can be recovered from regular ``solid diag'' output.

See~\ref{a:restart} for the details of the restart file
organization. The format of the restart files is the same as for
output files and is described in~\ref{a:bop}. It does not use external
libraries and generation of such files is trivial in any programming
language. Utility to convert the format to VTK are provided.
